% vim: set tabstop=8:softtabstop=4:shiftwidth=4:textwidth:78:formatoptions+=t% ===> this file was generated automatically by noweave --- better not edit it
\documentclass[11pt, a4paper]{article}
\usepackage{noweb}
\usepackage{amsmath}
\usepackage{a4wide}  % Set margins automatically for wider use of A4

% ----------
% Margins
% Left margin, odd pages: ("0.125" + 1)
\setlength{\oddsidemargin}{0.125in}
% Left margin, even pages: ("0.125" + 1)
\setlength{\evensidemargin}{0.125in}
% Text width 6.5 inch (so other margin is 1 inch).
\setlength{\textwidth}{6.125in}
% ----------

\title{Routines For Indoor Localization}
\author{EE712 Course Project}
\begin{document}
\maketitle

\nwfilename{ins.c.nw}\nwbegincode{1}\sublabel{NW469Jjs-To6Q8-1}\nwmargintag{{\nwtagstyle{}\subpageref{NW469Jjs-To6Q8-1}}}\moddef{boilerplate~{\nwtagstyle{}\subpageref{NW469Jjs-To6Q8-1}}}\endmoddef\nwstartdeflinemarkup\nwenddeflinemarkup
#include <stdio.h>
#include <stdlib.h>
#include <math.h>

// -----------------------------------------------------------------
// This package defines:
//
//    fnVectorNorm  : Calculates norm of a vector
//    fnDetectStill : Detect if the sensor is in still phase by
//                    comparing with threshold value
//    fnZuptVelocity: Estimate velocity by integrating the accelerometer
//                    readings
//
// -----------------------------------------------------------------

\LA{}global constants~{\nwtagstyle{}\subpageref{NW469Jjs-41E8u2-1}}\RA{}

\LA{}function fnVectorNorm~{\nwtagstyle{}\subpageref{NW469Jjs-uFbsF-1}}\RA{}

\LA{}function fnDetectStill~{\nwtagstyle{}\subpageref{NW469Jjs-Mrblu-1}}\RA{}

\LA{}function fnZuptVelocity~{\nwtagstyle{}\subpageref{NW469Jjs-QqJQw-1}}\RA{}

\nwnotused{boilerplate}\nwendcode{}\nwbegindocs{2}\nwdocspar
Routines to implement an indoor localization technique which use Inertial
Sensor (accelro-gyro-magneto). Adaptive Zero Velocity Update is the preferred
technique to correct the error which accumulates when integrating the
accelerometer output to compute velocity.

To apply adaptive zero velocity update, it is important to correctly detect
the still-phase of the foot and this is done by carefully setting the
threshold on the norm value of the gyro to detect zero-swing. This means
that the device has to be placed where there is some swing

Further, this threshold needs updating for different walking/running
speeds. The update algorithm is beyond the scope of this project. Further
as our device is designed for patients with limited mobility, it will be
sufficient to assume a swing on the lower end of walking speed (< 4kmph)

The steps to compute the velocity using ZUPT consists of the following:
\begin{enumerate}
\item Calculate norm (gyro)
\item Detect still-phase by comparing norm with threshold value
\item If not in still-phase, integrate accelerator output to get velocity
\item In in still-phase, velocity = 0
\item Integrate velocity to get position. This also includes figuring out
      the direction using the gyro and magneto.
\end{enumerate}

The {\Tt{}fnVectorNorm\nwendquote} function computes the norm of a vector. For example, 
\[
   \sqrt{\omega_{x,k}^2+\omega_{y,k}^2+\omega_{z,k}^2}
\],
   
where $\omega_{x,k}$ indicates the angular velocity along the $x$
coordinate at the $k^{th}$ sampling instance.

\nwenddocs{}\nwbegincode{3}\sublabel{NW469Jjs-uFbsF-1}\nwmargintag{{\nwtagstyle{}\subpageref{NW469Jjs-uFbsF-1}}}\moddef{function fnVectorNorm~{\nwtagstyle{}\subpageref{NW469Jjs-uFbsF-1}}}\endmoddef\nwstartdeflinemarkup\nwusesondefline{\\{NW469Jjs-To6Q8-1}}\nwenddeflinemarkup
// Computes the norm of a vector
float fnVectorNorm (float x, float y, float z) \{
   float xSq = x * x;
   float ySq = y * y;
   float zSq = z * z;

   return (sqrtf (xSq+ySq+zSq));
\}

\nwused{\\{NW469Jjs-To6Q8-1}}\nwendcode{}\nwbegindocs{4}\nwdocspar
The function {\Tt{}fnDetectStill\nwendquote} detects the still-phase by comparing the norm
of the angular velocity vector against a pre-determined threshold. The value
of this threshold depends on the walking/running speed of the subject.

\nwenddocs{}\nwbegincode{5}\sublabel{NW469Jjs-41E8u2-1}\nwmargintag{{\nwtagstyle{}\subpageref{NW469Jjs-41E8u2-1}}}\moddef{global constants~{\nwtagstyle{}\subpageref{NW469Jjs-41E8u2-1}}}\endmoddef\nwstartdeflinemarkup\nwusesondefline{\\{NW469Jjs-To6Q8-1}}\nwprevnextdefs{\relax}{NW469Jjs-41E8u2-2}\nwenddeflinemarkup
// threshold value to detect still phase (deg/sec)
float glStillPhaseThreshold = 0.01;

\nwalsodefined{\\{NW469Jjs-41E8u2-2}}\nwused{\\{NW469Jjs-To6Q8-1}}\nwendcode{}\nwbegincode{6}\sublabel{NW469Jjs-Mrblu-1}\nwmargintag{{\nwtagstyle{}\subpageref{NW469Jjs-Mrblu-1}}}\moddef{function fnDetectStill~{\nwtagstyle{}\subpageref{NW469Jjs-Mrblu-1}}}\endmoddef\nwstartdeflinemarkup\nwusesondefline{\\{NW469Jjs-To6Q8-1}}\nwenddeflinemarkup
// Detects still phase by comparing the gyro-norm against a threshold
int fnDetectStill (float gyroNorm) \{
   if (gyroNorm < glStillPhaseThreshold) return (1);
   else return (0);
   
\}

\nwused{\\{NW469Jjs-To6Q8-1}}\nwendcode{}\nwbegindocs{7}\nwdocspar
The function {\Tt{}fnZuptVelocity\nwendquote} implements the zero-update algorithm to
correct the velocity obtained from the accelerometer readings by using the
still-phase to zero the velocity. It returns a new velocity value.

\begin{itemize}
\item {\Tt{}ax,\ ay,\ az\nwendquote} are the accelerometer readings after gravity
corrections, and low-pass filtering
\item {\Tt{}gx,\ gy,\ gz\nwendquote} are the gyroscope readings after high-pass
filtering
\item {\Tt{}vt\nwendquote} is the current velocity
\end{itemize}

\nwenddocs{}\nwbegincode{8}\sublabel{NW469Jjs-41E8u2-2}\nwmargintag{{\nwtagstyle{}\subpageref{NW469Jjs-41E8u2-2}}}\moddef{global constants~{\nwtagstyle{}\subpageref{NW469Jjs-41E8u2-1}}}\plusendmoddef\nwstartdeflinemarkup\nwusesondefline{\\{NW469Jjs-To6Q8-1}}\nwprevnextdefs{NW469Jjs-41E8u2-1}{\relax}\nwenddeflinemarkup
// Accelerator sampling duration (seconds)
float glASamplingDuration = 0.001;

\nwused{\\{NW469Jjs-To6Q8-1}}\nwendcode{}\nwbegincode{9}\sublabel{NW469Jjs-QqJQw-1}\nwmargintag{{\nwtagstyle{}\subpageref{NW469Jjs-QqJQw-1}}}\moddef{function fnZuptVelocity~{\nwtagstyle{}\subpageref{NW469Jjs-QqJQw-1}}}\endmoddef\nwstartdeflinemarkup\nwusesondefline{\\{NW469Jjs-To6Q8-1}}\nwenddeflinemarkup
// Function to implement zero-update correction of the velocity values
// obtained by integrating the accelro output
float fnZuptVelocity (
      float ax, float ay, float az, float gx, float gy, float gz, float vt) \{

   float gyroNormValue = fnVectorNorm (gx, gy, gz);
   float vnew = vt;
   if (fnDetectStill (gyroNormValue) == 1) return (0.0);
   else \{
      vnew += glASamplingDuration * fnVectorNorm (ax, ay, az);
   \}

   return (vnew);
\}

\nwused{\\{NW469Jjs-To6Q8-1}}\nwendcode{}\nwbegindocs{10}\nwdocspar
Calculation of orientation of the body as it moves cannot be done by the gyro
readings alone as over time the error due to bias accumulates. This variation
in bias is due to the changing temperature of the gyroscope as it operates
(this is low frequency noise). 

\nwenddocs{}

\nwixlogsorted{c}{{boilerplate}{NW469Jjs-To6Q8-1}{\nwixd{NW469Jjs-To6Q8-1}}}%
\nwixlogsorted{c}{{function fnDetectStill}{NW469Jjs-Mrblu-1}{\nwixu{NW469Jjs-To6Q8-1}\nwixd{NW469Jjs-Mrblu-1}}}%
\nwixlogsorted{c}{{function fnVectorNorm}{NW469Jjs-uFbsF-1}{\nwixu{NW469Jjs-To6Q8-1}\nwixd{NW469Jjs-uFbsF-1}}}%
\nwixlogsorted{c}{{function fnZuptVelocity}{NW469Jjs-QqJQw-1}{\nwixu{NW469Jjs-To6Q8-1}\nwixd{NW469Jjs-QqJQw-1}}}%
\nwixlogsorted{c}{{global constants}{NW469Jjs-41E8u2-1}{\nwixu{NW469Jjs-To6Q8-1}\nwixd{NW469Jjs-41E8u2-1}\nwixd{NW469Jjs-41E8u2-2}}}%
\nwbegindocs{11}\nwdocspar
\end{document}
\nwenddocs{}
