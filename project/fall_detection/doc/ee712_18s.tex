\documentclass[11pt, a4paper]{article}
\usepackage[utf8]{inputenc}
\usepackage{a4wide}
\usepackage{anysize}
\usepackage[centertags]{amsmath}
\usepackage{amsfonts,amssymb,amsthm}
\usepackage{tikz}
\usetikzlibrary{positioning}
\usetikzlibrary{shapes}
\usepackage{graphicx}
\usepackage{placeins}   % to stop floats from floating
\usepackage[numbers]{natbib}    % so that natbib does not need author-year only numbers
\usepackage{wrapfig}
\usepackage{iitbieortitle}
\setcitestyle{square}
\renewcommand{\baselinestretch}{1.2} %line spacing
\marginsize{1.2in}{1.2in}{1in}{1in}  %left right top bottom
\title{Human Fall Detection Using Inertial Sensors}
\author{Name1, Name2 \and Niraj N Sharma (184077001)}
\begin{document}
\pagenumbering{arabic}
\maketitle
\section*{Abstract}

\section{Introduction}

\section{Literature Review}
Leading to the block diagram, ckt, components, etc.

Details from the paper: A survey on fall detection: Principles and approaches (Neurocomputing)
Details from the paper: (06091947.pdf) Optimum gravity vector and vertical acceleration estimation using a tri- axial accelerometer for falls and normal activities,” 
Details from the Hindawi paper

\section{Investigations}
any investigations carried out by you to develop a part of the design, verification of ideas, or other investigations.
\subsection{Accelerometer}
\subsubsection{Limitations}
\subsection{Gyroscope}
\subsubsection{Limitations}

\section{Final Block Diagram}

\section{Test Procedure}
to show how that the implementation achieves the requirement of the problem.

\section{Test Results}
with plots and tables (appendices, if necessary).

\section{Conclusion}
suggestions for further improvement.

\section{References}
\end{document}

