\documentclass[11pt, a4paper]{article}
\usepackage[utf8]{inputenc}
\usepackage{a4wide}
\usepackage{anysize}
\usepackage[centertags]{amsmath}
\usepackage{amsfonts,amssymb,amsthm}
\usepackage{tikz}
\usepackage[T1]{fontenc}
\usepackage{iitbieortitle}
\usetikzlibrary{positioning}
\usetikzlibrary{shapes}
\usepackage{graphicx}
\usepackage{placeins}   % to stop floats from floating
\usepackage[numbers]{natbib}    % so that natbib does not need author-year only numbers
\usepackage{wrapfig}
\usepackage{color}
\setcitestyle{square}
\renewcommand{\baselinestretch}{1.2} %line spacing
\marginsize{1.2in}{1.2in}{1in}{1in}  %left right top bottom

% Title Page
\begin{document}
\pagenumbering{roman}
\pagestyle{plain}
\def\title{Human Fall Detection}
\def\what{EE712 Embedded System Design}
\def\who{Simranjeet Singh (183076005) <183076005@iitb.ac.in> \\
Sonali Shukla (184070014) <184070014@iitb.ac.in> \\
Niraj N Sharma (184077001) <nirajns@iitb.ac.in> \\
Group 20}
\def\guide{Prof.~P.~C.~Pandey \& Prof.~D.~K.~Sharma}
\titlpage
\newpage

\pagenumbering{arabic}
\section*{Abstract}
In this project, we developed a wearable human fall-detection system
intended for the elderly. On detecting a fall, the system sends an
emergency SMS using the wearer's bluetooth enabled smartphone to a
caregiver. The system uses a GY-87 inertial sensor integrated with an
Arduino Nano micro-controller. The main objective of the project was to developed a low cost and small size wearable fall detection system with accurate detection of fall and direction of fall.

\section{Problem Statement}
Falls are a significant source of injury for the elderly. This wearable
device aims to detect a fall by detecting the change in accelerometer
output, before during and immediately after a fall. It also detects the
direction of the fall by combining readings from the accelerometer with
gyroscope output. The combination of readings from the two sensors using a
complementary filter results in more usable readings compared to
just the use of the accelerometer. The HC05 bluetooth module pairs the
wearable with the user's bluetooth enabled smartphone. In the event of a
fall the devices sends a message to the the smartphone which is loaded
with an application to convert the message into a SMS to a designated
caregiver.

\section{Literature Review}
\cite{Mubashir2013} is a comprehensive survey of recent fall detection
techniques which includes wearable based detection as well as detection
using ambient and vision sensors. Wearable based detection increasingly
gravitate towards accelerometer based techniques due to lower costs and
power efficiency of mems-based accelerometers. Accelerometer-based techniques
usually consist of detecting a sudden change in the acclerator output due to
the fall-event.

\noindent \cite{Bourke2011} discusses appropriate filtering techniques to
estimate the vertical acceleration vector from a tri-axial accelerometer. The
readings from a tri-axial accelerometer cannot be used directly as they tend
to be very noisy and are also combined with the acceleration due to the
earth's gravity.

\noindent \cite{Wu2015} uses a threshold based technique combining readings
from the accelerometer and gyroscope to detect the fall-event.

\section{Approach}
The idea is to use the I2C interface to interface a GY-87 sensor to the
Arduino Nano's micro-controller. The GY-87 provides separate readings for each
axes of the accelerator and gyroscope sensors. These six separate readings are
combined using a complementary filter. Our fall detection algorithm works on
the output of the complementary filter, by identifying a fall when the values
cross a certain preset threshold.

\noindent We then experimented with various regular movements like walking,
running (slowly), climbing stairs, sitting and sleeping to find an appropriate
threshold value below which a fall should not be detected. In addition to
comparing the combined accelro-gyro output against the threshold, we also use
declare that a fall has been detected only when the gyroscope indicates that
the orientation of the device has changed by at least 60 degrees.

\section{Hardware Setup}
Our hardware setup consists of four major components as show in
figure~\ref{fig:bd} on page~\pageref{fig:bd}.

\begin{description}
   \item [GY-87] The GY-87 also known as the MPU-60X0 integrates a
      tri-axial accelerometer and gyroscope with on-chip ADCs to deliver
      digital outputs via an I2C interface to an off-chip controller. To
      a controller this device acts as a slave on the I2C. GY-87 also
      features an auxiliary I2C bus which it masters to integrate
      discrete sensors like magnetometers.

   \item [Arduino Nano] The Arduino Nano is a board based on the
      ATmega328P microcontroller. The ATmega328 has 32 KB, (also with 2
      KB used for the bootloader). We have chosen the Nano due to its
      small form factor as the device is meant to be a wearable.

   \item [HC05] The HC05 is a bluetooth to serial port module which
      communicates via UART. The bluetooth module communicates fall
      information to a bluetooth enabled smartphone

   \item [Power Supply] A 9V battery is used as the power supply on the
      wearable.
\end{description}

\begin{figure}[htb]
    \centering
    \includegraphics[width=\linewidth]{BD.pdf}
    \caption{Block Diagram}
    \label{fig:bd}
\end{figure}


\subsection{I2C and UART Integration}
The I2C is used to integrate the GY-87 with the ATmega328P. The GY-87 behaves
as a slave, while the ATmega328P serves as the master. Figure
~\ref{fig:i2c_timing} on page~\pageref{fig:i2c_timing} describes a N-bit data
transfer on I2C.

\begin{figure}[htb]
    \centering
    \includegraphics[width=\linewidth]{I2C_Timing.pdf}
    \caption{I2C Data Transfer}
    \label{fig:i2c_timing}
\end{figure}

\noindent The HC05 communicates with via a UART interface which is run at a
baud rate of 9600 baud. Figure ~\ref{fig:uart_timing} on
page~\pageref{fig:uart_timing} describes a data transfer on the UART.

\begin{figure}[htb]
    \centering
    \includegraphics[width=\linewidth]{UART_Timing.pdf}
    \caption{UART Data Transfer}
    \label{fig:uart_timing}
\end{figure}

\subsection{GY87 Calibration and Integration}
Two experiments were conducted on the GY87 to calibrate it prior to
integration.

\subsubsection{Cube-mounted Measurements}
The GY-87 was mounted along the face of a thermocol cube. The cube itself was
placed on the surface of a flat table. The readings along the axis aligned to
the earth's gravitation field were verified to be as close to $\pm g$. This
experiment was repeated aligning each axis with g.

\subsubsection{Pendulum-mounted Measurements}
We attached the GY-87 to the arm of a pendulum and swung it, while measuring
the accelerometer output. As expected the output recorded its lowest value at
the point of the furthest swing, and was maximum near its rest position. The
figure~\ref{fig:pendulum} on page~\pageref{fig:pendulum} was our lab setup for
the experiment. 

\begin{figure}[htb]
    \centering
    \includegraphics[width=0.5\linewidth]{pendulum.jpg}
    \caption{Pendulum Experiment Assembly}
    \label{fig:pendulum}
\end{figure}

\subsection{Power Supply}
During development the assembly was powered using the micro-usb connection to
the Arduino-Nano. Once we assembled the device as a wearable, we were using a
9V cell with a capacity of 400 mAH to power the device. The following table
lists the different devices and their peak current consumption:

\begin{center}
   \begin{tabular}{llll}
      \hline
      Device & Current Drawn (mA) \\
      \hline
      Arduino Nano & $2.4$ \\
      GY-87 Sensors & $3.9$ \\
      LEDs (4) & $16.0$ & \\
      HC-05 & $30.0$ \\
      \hline
      Total & $52.3$ \\
      \hline
   \end{tabular}
\end{center}

\noindent The device can draw a peak current of $52.3$ mA. At this
consumption, a 9V cell will last for about $7.50$ hours.

\section{Software}
The figure~\ref{fig:sw_flow} on page~\pageref{fig:sw_flow} describes the
processing which occurs with the accelerometer and gyroscope sensor readings
from the GY-87. The input from the two sensors are received as a set of six,
16-bit values, one for each axis of each sensor. \\

To detect the peak value of that trigger the further algorithm is the magnitude of the accelerometer data on all direction(x-axis, y-axis, x-axis)
\begin{equation}
	\label{eq:mag}
	Magnitude = \sqrt{x^{2} + y^{2} + z^{2}}
\end{equation}
The value of magnitude will trigger the further code to check the direction on which side actual fall has happen. The threshold for magnitude peak we chosen was 180. If the $magnitute \geq 180$ then roll and pitch calculated from the complementary filter are used to detect the fall direction. The roll and pitch value is changing from 90. We have chosen the threshold as 60.

\noindent The accelerometer values are taken through a LPF, while the
gyroscope values are taken through a HPF. These filtered values are then
combined using a complementary filter which essentially does a weighted sum of the sensor values.
\begin{figure}[htb]
    \centering
    \includegraphics[width=\linewidth]{SW_Flow.pdf}
    \caption{Software Flow}
    \label{fig:sw_flow}
\end{figure}

\section{Final Assembly}
The PCB which integrated the GY87, the Arduino Nano and the HC-05 bluetooth module have been housed in a custom-made acrylic box with a separate compartments for the battery. Cutouts in the box allows us to access the switch and the USB debug interface. This box was attached to a frame which served as the human torso. Dropping the frame from a standing position would mimic a fall.

\begin{figure}
    \centering
    \includegraphics[width=\linewidth]{Box.png}
    \caption{Device Assembly}
    \label{fig:box}
\end{figure}

\section{Test Procedure and Results}
We tested the device both with normal movements, such as walking, sitting,
jumping and sleeping. The tests also included transitions between these
movements as the transition points were more interesting from our
application's point-of-view. The following are plots of everyday actions
like jogging, walking, sitting, sleeping.

\noindent In all these plots, the blue plot indicates filtered accelerometer
output. The red and green plots indicate roll and pitch.

\noindent In the walk test, the wearer walked at a normal pace during this
test. The plot is on figure~\ref{fig:walk} on page~\pageref{fig:walk}.
\begin{figure}
    \centering
    \includegraphics[width=0.7\linewidth]{walk.png}
    \caption{Walking}
    \label{fig:walk}
\end{figure}

\noindent The sit-stand test was conducted when the wearer sat into a short chair from a standing position and stood up again. The plot is on figure~\ref{fig:ss} on page~\pageref{fig:ss}.
\begin{figure}
    \centering
    \includegraphics[width=0.7\linewidth]{sit-stand.png}
    \caption{Sitting-Standing}
    \label{fig:ss}
\end{figure}

\noindent The jump test was conducted when the wearer was jumping in place. This action most closely resembles a fall, and the test was conducted to ensure our thresholds was correct. The plot is on figure~\ref{fig:jumping} on page ~\pageref{fig:jumping}.
\begin{figure}
    \centering
    \includegraphics[width=0.70\linewidth]{jump.png}
    \caption{Jumping}
    \label{fig:jumping}
\end{figure}

\noindent The sleep-and-walk test was conducted when the wearer was lying down and suddenly
got up. The plot is on figure~\ref{fig:sleeping} on page ~\pageref{fig:sleeping}.
\begin{figure}
    \centering
    \includegraphics[width=0.7\linewidth]{sleep.png}
    \caption{Sleeping-Waking}
    \label{fig:sleeping}
\end{figure}

\section{Conclusion}
We have implemented a device which detects if the wearer has had a fall, and
classifies the fall direction -- front, back, right or left. This information
is communicated via Bluetooth to the wearer's smartphone.

\noindent A limitation of this device is that orientation of the device has
to be fixed with respect to the wearer.  A future improvement could be for the
device to automatically calibrate its current orientation with respect to a
preset reference. This would allow the wearer greater flexibility in device
placement.

\bibliographystyle{ieeetr}
\bibliography{reference-papers}
\end{document}

